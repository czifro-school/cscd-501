\documentclass[11pt]{article}
\usepackage{fullpage}
\usepackage{graphics,epsfig,color}
\usepackage{wrapfig}

\usepackage{times}
\usepackage{setspace}
\usepackage{amsmath,amsthm,amssymb}
%\usepackage[ruled,vlined,linesnumbered]{algorithm2e}
\usepackage{qtree}
\usepackage{subfigure}
\usepackage{url}


%for code from latexdraw
%\usepackage[usenames,dvipsnames]{pstricks}
%\usepackage{epsfig}
%\usepackage{pst-grad} % For gradients
%\usepackage{pst-plot} % For axes


\newtheorem{theorem}{Theorem}[section]
%\newtheorem{proposition}{Proposition}[theorem]
\newtheorem{corollary}{Corollary}[section]
\newtheorem{lemma}{Lemma}[section]
%\newtheorem{claim}{Claim}[section]
\newtheorem{problem}{Problem}
%\newtheorem{conjecture}{Conjecture}[section]
\newtheorem{definition}{Definition}[section]
\newtheorem{observation}{Observation}[section]
\newtheorem{example}{Example}[section]
\newtheorem{openproblem}{Open Problem}[section]
\newtheorem{fact}{Fact}[section]
%\newcommand{\qedsymb}{\hfill{\rule{2mm}{2mm}}}

\newcommand{\qedsymb}{\hfill{\rule{2mm}{2mm}}}
\newenvironment{proofsketch}{\begin{trivlist}
\item[\hspace{\labelsep}{\noindent Proof Sketch: }]
}{\qedsymb\end{trivlist}}



%the following few lines until usepackage{algorithm2e} is to avoid the
%conflicts of algorithm2e with other packages.
\makeatletter
\newif\if@restonecol
\makeatother
\let\algorithm\relax
\let\endalgorithm\relax
%\usepackage[ruled,vlined,linesnumbered]{algorithm2e}
\usepackage[ruled,vlined]{algorithm2e}


%\newenvironment{proof}{\begin{trivlist}
%\item[\hspace{\labelsep}{\bf\noindent Proof: }]}{\qedsymb\end{trivlist}}
%\newcommand{\qed}{\hfill\rule{2mm}{2mm}}

\newcommand{\remove}[1]{}



%--------------------------------


\begin{document}

\begin{center}
  {\LARGE CSCD501 Homework2}

\bigskip 

{\Large Will Czifro}

\end{center}

\bigskip 

\begin{problem}[100 points]
\label{prob:1}
Given an arbitrary permutation of the set {1, 2,...,n} stored in an array of size n, replace
it with the next permutation in lexicographical order. For example, when n = 3, here is a sequence of all the permutations
listed in lexicographical order: (1, 2, 3),(1, 3, 2),(2, 1, 3),(2, 3, 1),(3, 1, 2),(3, 2, 1). As a special case, if
the current permutation is (n, n − 1,..., 1), then the next permutation should be (1, 2,...,n), such that the sequence
of permutations is cyclical. Assume your permutation generation algorithm will be executed n! consecutive times, so
that the final permutation will be the same as the original permutation.

\begin{enumerate}
  \item (10 points) Write an efficient deterministic algorithm:

    \begin{enumerate}
      \item Input: a permutation of the set {1, 2,...,n}
      \item Output: the next permutation in lexicographical order.
    \end{enumerate}

  \item (10 points) Analyze the worst-case running time of your algorithm.
  \item (10 points) Provide an example that shows when the worst-case time can occur.
  \item (20 points) Analyze the amortized time cost of your algorithm using the aggregate method.
  \item (20 points) Analyze the amortized time cost of your algorithm using the accounting method.
  \item (30 points) Analyze the amortized time cost of your algorithm using the potential method.
\end{enumerate}

Hint: The amortized time cost should be significantly better than the worst-case time, otherwise you should either
improve your algorithm or tighten the analysis.
\end{problem}

%\end{document}

\newpage

%---------------------------------------

\bigskip
\noindent{\bf Solution for Problem~\ref{prob:1}.}
  
%\IncMargin{1em}
\begin{algorithm}
\DontPrintSemicolon
\SetKwInOut{Input}{input}\SetKwInOut{Output}{output}
\SetKw{KwTo}{in}\SetKw{KwDownTo}{downto}\SetKw{Break}{break}
\SetKwFunction{abs}{abs}
\Input{integer set $A$}
\Output{$A$ after one lexicographical permutation}
\BlankLine
$n \leftarrow A.length$

$x \leftarrow -1$
\tcp*[h]{This will contain index we need to swap}\label{cmt}

$y \leftarrow -1$
\tcp*[h]{This will contain index we need to swap with}\label{cmt}
\BlankLine
\tcp{find first instance where A[i+1] > A[i]}\label{cmt}
\For{$i$ \KwTo $n-1$ \KwDownTo $0$}{
  \tcp{A[n-1] has no successor}\label{cmt}
  \If{$i+1 \leq n-1$}{
    \If{$A[i] < A[i+1]$}{
      $x \leftarrow i$ \;
      \Break
    }
  }
}
\BlankLine
\tcp{need to find index, y, of least greater value where y > x}\label{cmt}
$minDiff \leftarrow -1$ \;
$y \leftarrow n-1$ \;
\For{$i$ \KwTo $n-1$ \KwDownTo $max(x,0)$}{
  \If{$A[i] > A[max(x,0)]$}{
    \If{$minDiff = -1$ or $minDiff > A[i] - A[max(x,0)]$}{
      $minDiff \leftarrow A[i] - A[max(x,0)]$ \;
      $y \leftarrow i$ 
  }
  }
}
\tcp{swap values at specified indices if x was found}\label{cmt}
\If{$x \neq -1$}{$swap(A,x,y)$}
\BlankLine
\tcp{everything past x needs to be lexicographically ordered}\label{cmt}
$i \leftarrow x + 1$ \;
$j \leftarrow n - 1$ \;
\While{$i \neq j$ and $i < j$}{
  \tcp{swap values at specified indices}\label{cmt}
  $swap(A,i,j)$
}

\Return{A}

\caption{nextPermutation(A)}\label{algo_disjdecomp}
\end{algorithm}%\DecMargin{1em}


\begin{enumerate}
\item above pseudocode
\item The algorithm starts out by reverse scanning for the largest $ith$ position that meets the following condition $A[i] < A[i+1]$. This $ith$ position is referred to as the $pivot$. Finding the $pivot$ in the worst case requires a full scan that results in $pivot \leftarrow 0$, which is $\Theta(n)$. Once the $pivot$ is found, the $kth$ value, referred to as the $swap$, will be reverse scanned for that meets the following condition: $k > i$ and $A[k] > A[i]$ such that of the possible values, $A[k] - A[i]$ is the smallest possible difference. Given these conditions, there exists a worst case where there are no values past the $ith$ position that satisfy these conditions. In this case, this step will take $\Theta(n)$ time. After this the $pivot$ and $swap$ are swapped in the set. This, however, can only happen if a $pivot$ was found. Otherwise it is skipped. The last step is to make sure the set in in lexicographical order for the permutation. Since the relationship between the $pivot$ and the $swap$ is that $A[i] < A[k]$ provided $i < k$, it can be stated that the sub array $A[i+1..n-1]$ is in descending order. Furthermore, the following can be stated: $A[i+1..k] > A[i]$ \&\& $A[k+1..n-1] < A[i]$. Given this, after the swap occurs, the sub array $A[i+1..n-1]$ will remain in descending order. Thus, forwards iteration and backwards iteration can be done simultaneously swapping values each step until both iterators meet in the middle resulting in a lexicographically ordered set. This will at most, swap $\frac{n}{2}$ values, which means the worst case will take $\Theta(\frac{n}{2})$. So for the entire function, the time cost is $T(2n + \frac{n}{2})$, or simply, $T(3n)$ at most. 
\item For the worst case scenario, consider the following set:

$$
\{ 10, 9, 8, 7, 6, 5, 4, 3, 2, 1 \}
$$

In this set, a $pivot$ won't be found, which defaults to $pivot \leftarrow 0$. Scanning for a $swap$ will iterate through the entire list and not find a suitable value defaulting to $swap \leftarrow n-1$. Since a pivot could not be found, no swapping occurs between $pivot$ and $swap$. The last step then reorders the entire set, which requires the full $\frac{n}{2}$ operations. Afterwards, the set results in the following:

$$
\{ 1, 2, 3, 4, 5, 6, 7, 8, 9, 10 \}
$$

\item 

\begin{proof}
Need to prove that the amortized cost of this algorithm is $O(1)$ using the aggregate method.

Consider the following permutations from the set $\{ 3, 4 \}$, where $n = 2$:

$$
[3], 4
$$
$$
4, 3
$$

Now consider the following permutations from the set $\{ 2, 3, 4 \}$, where $n = 3$:

$$
2, [3], 4
$$
$$
[2], 4, 3
$$
$$
3, [2], 4
$$
$$
[3], 4, 2
$$
$$
4, [2], 3
$$
$$
4, 3, 2
$$

The bracketed values are the pivots used to get the next permutation. Let $k$ represent the number of permutations, $k = n!$. In the first set, $k$ is $2!$ an in the second set, $k$ is $3!$. 

The set $n = 2$ had two permutations, one of which had a $pivot$ and a sub array of length 1. With the second set, there were 6 permutations, 3 of which had a $pivot$ where the sub array was of length 1. There were 2 permutations that had a $pivot$ with a sub array of length 2. When $i = 1$ in both of the above sets, the frequency is $\frac{1}{2}$. When $i = 2$ in the above set, the frequency is $\frac{2}{6}$. If $n = 4$, it would be observed that the frequency of $i = 3$ is $\frac{3}{24}$. So, it can be stated that the frequency of a sub array of length $i$ is $\frac{i}{(i+1)!}$.

It should be noted that the cost of a finding the $jth$ permutation is $c_j = 3i_j$. With this, the cost of finding $m$ permutations is as follows:

$$
\sum_{i=1}^{k} c_i = \sum_{i=1}^{m} 3ik \cdot \frac{i}{(i+1)!}
$$
$$
 = 3k \sum_{i=1}^{m} \frac{i^2}{(i+1)!}
$$
$$
 = 3k \cdot (e-1)
$$
$$
 = 3k \cdot 2 = 6k
$$

The cost of doing $k$ permutations is $6k$. Therefore, the cost for a single permutation is 6 thus making the time complexity $O(1)$.


\end{proof}

\item

\begin{proof}
Need to prove that the amortized cost of this algorithm is $O(1)$ using the accounting method.

To start, let's define the operations that need to be payed for. Every function call requires finding the $pivot$ and $swap$, swapping both values, and reversing the sequence after the $pivot$. To find the $pivot$, the cost is $x$, where $x$ is the number of iterations needed to find the $pivot$. To find the $swap$, the cost is $x-1$. To do the swap is contant time of 1. Reversing the sequence after the $pivot$ costs $\lfloor\frac{x}{2}\rfloor$. So for each step, the cost is $x +$ $(x-1) +$ $\lfloor\frac{x}{2}\rfloor + 1$. Take the example of a sequence of length 4, represented by $n$, and the number of permutations is $k = n!$. For $k$ permutations, the actual cost was 143. To get the amount that needs to be invested each permutation, the following equation needs to hold true:

$$
\sum_{i=1}^{k}\hat{c_i} \geq \sum_{i=1}^{k}c_i
$$

Given this equation, the only $\hat{c}$ that can satisfy this is 6 because $6k = 144$. With this, each permutation gets an investment of $\$6$ every time a $pivot$ is found. \$x is paid to find the pivot and $\$(x-1)$ is paid to find the $swap$ and $\$1$ is paid to do the swap. $\$\lfloor\frac{x}{2}\rfloor$ is paid to reverse the sequence after the $pivot$. There will be instances where an investment will not cover the cost of a function call. In this instance, borrowing credit will need to happen. The amortized cost, then, is $\Theta(6k) / k = \Theta(6) = \Theta(1)$.

\end{proof}

\newpage

\item

\begin{proof}
Need to prove that the amortized cost of this algorithm is $O(1)$ using the potential method.

A $\phi$ function needs to be defined such that the following equation for a single step holds true:

$$
\hat{c_i} = c_i + \phi_i - \phi_{i-1}
$$

For the $ith$ operation were the permutation results in generated pairs of descending digits, the following $\phi$ function is used

$$
\phi_i = 2x + y
$$

where $x$ is the number of pairs in descending order created and $y$ is the number of pre-existing descending pairs.

For the $ith$ operation where the permutation includes a reversal, the following $\phi$ function is used:

$$
\phi_i = x
$$

So, for descending pair creation:

$$
\hat{c_i} = c_i + \phi_i - \phi_{i-1}
$$
$$
\hat{c_i} = c_i + 2x_i + y_i - (2x_{i-1} + y_{i-1})
$$
$$
\hat{c_i} = c_i + 2x_i + y_i - (2x_i + y_i - 2)
$$
$$
\hat{c_3} = c_3 + 2(1) + 1 - (2(1) - 1)
$$
$$
\hat{c_i} = 4 + 2(1) + 1 - (2(1) - 1)
$$
$$
\hat{c_i} = 6
$$

This shows the amortized cost being $O(1)$.

\end{proof}

\end{enumerate}

%---------------------------------------


\end{document}




