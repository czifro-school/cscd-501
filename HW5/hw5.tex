\documentclass[11pt]{article}
\usepackage{fullpage}
\usepackage{graphics,epsfig,color}
\usepackage{wrapfig}

\usepackage{times}
\usepackage{setspace}
\usepackage{amsmath,amsthm,amssymb}
%\usepackage[ruled,vlined,linesnumbered]{algorithm2e}
\usepackage{qtree}
\usepackage{subfigure}
\usepackage{url}



%for code from latexdraw
%\usepackage[usenames,dvipsnames]{pstricks}
%\usepackage{epsfig}
%\usepackage{pst-grad} % For gradients
%\usepackage{pst-plot} % For axes


\newtheorem{theorem}{Theorem}[section]
%\newtheorem{proposition}{Proposition}[theorem]
\newtheorem{corollary}{Corollary}[section]
\newtheorem{lemma}{Lemma}[section]
%\newtheorem{claim}{Claim}[section]
\newtheorem{problem}{Problem}
%\newtheorem{conjecture}{Conjecture}[section]
\newtheorem{definition}{Definition}[section]
\newtheorem{observation}{Observation}[section]
\newtheorem{example}{Example}[section]
\newtheorem{openproblem}{Open Problem}[section]
\newtheorem{fact}{Fact}[section]
%\newcommand{\qedsymb}{\hfill{\rule{2mm}{2mm}}}

\newcommand{\qedsymb}{\hfill{\rule{2mm}{2mm}}}
\newenvironment{proofsketch}{\begin{trivlist}
\item[\hspace{\labelsep}{\noindent Proof Sketch: }]
}{\qedsymb\end{trivlist}}



%the following few lines until usepackage{algorithm2e} is to avoid the
%conflicts of algorithm2e with other packages.
\makeatletter
\newif\if@restonecol
\makeatother
\let\algorithm\relax
\let\endalgorithm\relax
%\usepackage[ruled,vlined,linesnumbered]{algorithm2e}
\usepackage[ruled,vlined]{algorithm2e}


%\newenvironment{proof}{\begin{trivlist}
%\item[\hspace{\labelsep}{\bf\noindent Proof: }]}{\qedsymb\end{trivlist}}
%\newcommand{\qed}{\hfill\rule{2mm}{2mm}}

\newcommand{\remove}[1]{}



%--------------------------------


\begin{document}

\begin{center}
  {\LARGE CSCD501 Homework5}

\bigskip 

{\Large Will Czifro}

\end{center}

\bigskip 

\begin{problem}[25 points]
\label{prob:1}
Trace the DFT of the vector (1, 2, 3, 4). That is, show the DFT’s of the all the recursive calls
in computing the DFT of the given vector that uses $\omega_4^i$ as the points, where i = 0, 1, 2, 3.
\end{problem}



%---------------------------------------

\begin{problem}[25 points]
\label{prob:2}
We define the distance of two points $p_1 = (x_1, y_1)$ and $p_2 = (x_2, y_2)$ as

$$
\max{\{|x_1 - x_2|,|y_1 - y_2|\}}
$$

Given a set of n points in the 2-d space with their x and y coordinates. Design an efficient algorithm for finding
the closest pair of points and their distance. Explain your algorithmic idea and its time complexity using the big-oh
notation. (Hint: modify the algorithm for finding the closest pair of points that we discussed in the lecture.)
\end{problem}



%---------------------------------------

\begin{problem}[25 points]
\label{prob:3}
Suppose $X \leq_P Y$. Determine the BEST choice for the following questions.
\begin{itemize}
\item If $X \in P$, then: 1) $Y \in P$ 2) $Y \in NP\textendash hard$ 3) Both 4) Neither
\item If $X \in NP\textendash hard$, then: 1) $Y \in P$ 2) $Y \in NP\textendash hard$ 3) Both 4) Neither
\item If $Y \in P$, then: 1) $X \in P$ 2) $X \in NP\textendash hard$ 3) Both 4) Neither
\item If $Y \in NP\textendash hard$, then: 1) $X \in P$ 2) $X \in NP\textendash hard$ 3) Both 4) Neither
\item If $X \in P$ and $Y \in NP\textendash hard$, then: 1) $P = NP$ 2) $P \neq NP$ 3) Both 4) Neither
\item If $X \in NP\textendash hard$ and $Y \in P$, then: 1) $P = NP$ 2) $P \neq NP$ 3) Both 4) Neither
\end{itemize}
\end{problem}




\begin{problem}[25 points]
\label{prob:4}
Show the polynomial-time reduction between the following two problems $X$ and $Y$ . That is,
explain how to solve $X$ by using a blackbox that solves $Y$ , and the vice versa.
\begin{itemize}
\item Problem X \\
Given a graph $G = (V,E)$ and its three vertices $a, b, c \in V$ , determine whether $G$ has a simple path from $a$ to $c$ that goes through $b$ ?
\item Problem Y \\
Given a graph $G = (V,E)$ and its two vertices $d, e \in V$ , determine whether $G$ has a simple cycle that goes through both $d$ and $e$.
\end{itemize}
\end{problem}


\newpage

%---------------------------------------

\bigskip
\noindent{\bf Solution for Problem~\ref{prob:1}.}

Trace through DFT for vector $<1,2,3,4>$:
\\
First let's do the recursive steps that get us down to one element
\begin{equation}
\begin{split}
& Let A = <1,2,3,4>\\
\\
& A(x) = A^{[0]}(x^2) + x A^{[1]}(x^2)\\
\\
& A(\omega_4) = 1\omega_4^0 + 2\omega_4^1 + 3\omega_4^2 + 4\omega_4^3\\
& \quad A^{[0]}(\omega_2) = 1\omega_2^0 + 3\omega_2^1\\
& \quad \quad A^{[0][0]}(\omega_1) = 1\omega_1^0\\
& \quad \quad A^{[0][1]}(\omega_1) = 3\omega_1^0\\
& \quad A^{[1]}(\omega_2) = 2\omega_2^0 + 4\omega_2^1\\
& \quad \quad A^{[1][0]}(\omega_1) = 2\omega_1^0\\
& \quad \quad A^{[1][1]}(\omega_1) = 4\omega_1^0
\end{split}
\end{equation}

Now that we have single element vectors, let's work back up the chain
merging results together starting at $A^{[0]}$

\begin{equation}
\begin{split}
& \quad \quad A^{[0]} = <1 + (\omega_2^0\cdot 3), 1 + (\omega_2^1\cdot 3)> = <4,-2>\\
& \quad \quad A^{[1]} = <2 + (\omega_2^0\cdot 4), 2 + (\omega_2^1\cdot 4)> = <6,-2>\\
& \quad A = <4 + (\omega_4^0\cdot 6), -2 + (\omega_4^1\cdot -2), 4 + (\omega_4^2\cdot 6), -2 + (\omega_4^3\cdot -2)>\\
& \quad \;\;\; = <10, -2 - 2i, -2, -2 + 2i>
\end{split}
\end{equation}

The original vector has now turned into $<10, -2 - 2i, -2, -2 + 2i>$

%---------------------------------------
\bigskip

\noindent{\bf Solution for Problem~\ref{prob:2}.}

Consider Algorithms 1 and 2. The time complexity is $O(n\log n)$ for this algorithm.


%---------------------------------------
\bigskip

\noindent{\bf Solution for Problem~\ref{prob:3}.}

\begin{itemize}
\item If $X \in P$, then Neither\\
Given what is known about $P$ and $NP$ and the fact we can only say $X \in P$, we cannot prove that $Y \in P$ and/or $Y \in NP\textendash hard$
\item If $X \in NP\textendash hard$, then $Y \in NP\textendash hard$\\
The only way we can have $X \leq_P Y$ and $X \in NP\textendash hard$ is if we have $Y \in NP\textendash hard$. Since there is no proof that $P = NP$ (or $P \neq NP$), we cannot say anything more about $Y$
\item If $Y \in P$, then $X \in P$\\
We can make this claim because there is no proof that $P = NP$. Any $P$ function can be reduced to another $P$ function.
\item If $Y \in NP\textendash hard$, then Neither\\
There is insufficient evidence to say anything about $X$, so it is neither
\item If $X \in P$ and $Y \in NP\textendash hard$, then Neither\\
There is insufficient evidence to make the claim $P = NP$ or $P \neq NP$, so it is neither
\item If $X \in NP\textendash hard$ and $Y \in P$, then $P = NP$\\
If we have $X \in NP\textendash hard$ and $Y \in P$, we are reducing a $NP\textendash hard$ problem to be no harder than a $P$ problem implying that $P = NP$.
\end{itemize}


%---------------------------------------
\bigskip

\noindent{\bf Solution for Problem~\ref{prob:4}.}

\begin{itemize}
\item Problem X:\\
For the function $X$ we have the parameters $G, a, c, b$. This is to read as follows: in directed graph $G$, is there a simple path from $a$ to $c$ that passes through $b$. Solving this will require blackbox function $Y$ with parameters $G, d, e$, which read as follows: in directed graph $G$, is there a simple cycle that passes through $d$ and $e$. If we are limited to using only $Y$ to solve this, we will need to augment $G$ so that the boolean values returned by $Y$ can confirm or deny that there exists a simple path from $a$ to $c$, denoted as $a\rightarrow c$, that passes through $b$, denoted as $a\rightarrow b\rightarrow c$. A simple way to do this would be to add a new vertex to $G.V$. If we add $a^\prime$ to $G.V$ and add $(a^\prime,a)$ and $(c,a^\prime)$ to $G.E$. We then can check if this added point is in a cycle with $b$ because the only way $a^\prime$ can reach $b$ would be through $a$, and $b$ could only reach $a^\prime$ by passing through $c$. This proves the existence of a path $a\rightarrow b\rightarrow c$. This works for undirected graphs if we view it as bi directional and solve it for one direction. Consider Algorithm 3.
\item Problem Y:\\
For the function $Y$ we have the parameters $G, d, e$. This is to read as follows: in directed graph $G$, is there a simple cycle that passes through $d$ and $e$. Solving this will require blackbox function $X$ with parameters $G, a, c, b$, which read as follows: in directed graph $G$, is there a simple path from $a$ to $c$ that passes through $b$. If we are limited to using only $X$ to solve this, we will need to augment $G$ so that the boolean values returned by $X$ can confirm or deny that there exists a simple cycle that passes through $d$ and $e$, denoted as $d \leftrightarrow e$. In order to use $X$ to confirm or deny $d \leftrightarrow e$, we need to transform that cycle into a path problem. To do that, we need to look at $d \leftrightarrow e $ as the composition of simple paths $d \rightarrow e$ and $e \rightarrow d$. If we take the edge going into $d$ in $e \rightarrow d$, $(d^\prime,d)$, and remove it, $e \rightarrow d$ no longer exists and we have also eliminated the cycle because of our composition. However, we would now have a simple path $d \rightarrow d^\prime$ that passes through $e$, or $d \rightarrow e \rightarrow d^\prime$. This is something $X$ can confirm for us. So, after removing $(d^\prime,d)$ from $G.E$, if $X(G,d,d^\prime,e)$ returns true, we have proven that $d \leftrightarrow e$ exists. Do for all edges going into $d$. This works for undirected graphs if we view it as bi directional and solve it for one direction. Consider Algorithm 4.
\end{itemize}

%---------------------------------------
\newpage

\begin{algorithm}
\DontPrintSemicolon
\SetKwInOut{Input}{input}\SetKwInOut{Output}{output}
\SetKw{KwTo}{t}
\Input{Q is a set of n points in the 2-d space\\
       Each point has (x,y) coordinates}
\Output{One closest pair of points in Q and their distance}
\BlankLine
\If{$|Q| \leq 3$} {
  find and return the closest pair and their distance using a naive method
}
\;
\tcp{use linear-time algorithm to find median x-coordinate. Time Cost: $O(n)$}\label{cmt}
\;
Vertically cut Q into two (almost) evenly-sized pieces: L and R,
with the order of y-coordinates preserved in both L and R.
\;
$(pt1_L, pt2_L, delta_L) \leftarrow ClosestPair(L)$ \tcp*{Time Cost: $T(n/2)$}
$(pt1_R, pt2_R, delta_R) \leftarrow ClosestPair(R)$ \tcp*{Time Cost: $T(n/2)$}
$delta \leftarrow \min(delta_L, delta_R)$\;
$(pt1_{span}, pt2_{span}, delta_{span}) \leftarrow ClosestPairSpan(Q,delta)$ \tcp*{Time Cost: $O(n)$}

\If{$delta < delta_{span}$}{
  \If{$delta = delta_L$}{ \Return{$(pt1_L, pt2_L, delta_L)$} }
  \Return{$(pt1_R, pt2_R, delta_R)$}
}
\Return{$(pt1_{span}, pt2_{span}, delta_{span})$}
\caption{ClosestPair(Q)}\label{algo_closestpair}
\end{algorithm}

\begin{algorithm}
\DontPrintSemicolon
\SetKwInOut{Input}{input}\SetKwInOut{Output}{output}
\SetKw{KwTo}{t}
\Input{Q is a set of n points in the 2-d space\\
       Each point has (x,y) coordinates\\
       delta is for getting points within delta of the median}
\Output{One closest pair of points in Q and their distance}
\BlankLine
Collect all the points from Q that are within the 2*delta strip with the order of y-coordnates preserved among the collected points \tcp*{Time: $O(n)$}
\;
Let $<p_1\ldots p_m>$ be the collected points. \tcp*{$m \leq n$}
\;
$delta_{span} \leftarrow infinity$\;
$pt1_{span} \leftarrow null$\;
$pt2_{span} \leftarrow null$\;
\;
\tcp{Time: $O(m) = O(n)$}
\For{$i = 1 \ldots m-1$}
{
  \For{$j = i+1 \ldots \min(i+7,m)$}{
    \If{$delta_{span} > \max(|p_i.X-p_j.X|,|p_i.Y-p_j.Y|)$}{
      $delta_{span} \leftarrow \max(|p_i.X-p_j.X|,|p_i.Y-p_j.Y|)$\;
      $pt1_{span} \leftarrow p_i$\;
      $pt2_{span} \leftarrow p_j$\;
    }
  }
}
\Return{$(pt1_{span}, pt2_{span}, delta_{span})$}
\caption{ClosestPairSpan(Q,delta)}\label{algo_closestpairspan}
\end{algorithm}

\begin{algorithm}
\DontPrintSemicolon
\SetKwInOut{Input}{input}\SetKwInOut{Output}{output}
\SetKw{KwTo}{t}
\Input{G is a directed graph\\
       a is the beginning of a simple path\\
       c is the end of a simple path\\
       b is the point that lies on the simple path}
\Output{A boolean stating whether or not there is a 
        simple path from a to c with point b}
\BlankLine
Add point $a^\prime$ to $G.V$\;
Add edges $(a^\prime,a)$ and $(c,a^\prime)$ to $G.E$\;
$isNowCycle \leftarrow Y(G,a^\prime,b)$\;
Remove edges $(a^\prime,a)$ and $(c,a^\prime)$ to $G.E$ \tcp{maintain graph integrity}
Remove point $a^\prime$ to $G.V$ \tcp{maintain graph integrity}
\Return{isNowCycle}
\caption{X(G,a,c,b)}\label{algo_X}
\end{algorithm}

\begin{algorithm}
\DontPrintSemicolon
\SetKwInOut{Input}{input}\SetKwInOut{Output}{output}
\SetKw{KwTo}{t}
\Input{G is a directed graph\\
       d is a point in a simple cycle\\
       e is a point in a simple cycle}
\Output{A boolean stating whether or not there is a 
        simple cycle with d and e}
\BlankLine
\tcp{See if we can turn the cycle containing d and e}
\tcp{into a simple starting at d to some node just before d}
\tcp{that passes through e using function X}
Let Z be all incoming edges into $d$\;
\For{$(d^\prime,d) \in Z$}{
  Remove $(d^\prime,d)$ from $G.E$
  \If{$X(G,d,d^\prime,e) = true$}{
    \tcp{We found the edge that completes the cycle}
    Add $(d^\prime,d)$ to $G.E$ \tcp{maintain graph integrity}
    \Return{true}
  }
  Add $(d^\prime,d)$ to $G.E$ \tcp{maintain graph integrity}
}
\tcp{There is no cycle with d and e}
\Return{false}
\caption{Y(G,d,e)}\label{algo_Y}
\end{algorithm}

%---------------------------------------

\end{document}




